\chapter{Kapitel im Inhaltsverzeichnis}
Dieses Kapitel ist im Inhaltsverzeichnis sichtbar.

\chapter*{Kapitel nicht im Inhaltsverzeichnis}
Dieses Kapitel ist nicht im Inhaltsverzeichnis sichtbar und wird nicht nummeriert.

\chapter{Text schreiben}
Text wird einfach geschrieben.
Er kann auch \textbf{fett} oder \textit{kursiv} oder \textit{\textbf{beides}} sein.

Sogar \textcolor{blue}{die Farbe} \textcolor{red}{kann} \textcolor{magenta}{geändert} werden!

Mit einer Leerzeilen wird ein Absatz erstellt.
Mit \anfz{\textbackslash\textbackslash} wird ein\\
Zeilenumbruch erstellt.

Ganz einfach per \anfz{\textbackslash{}anfz\{Text\}} wird der Text in Anführungszeichen geschrieben.

Quellen werden mit einem \anfz{\textbackslash{}cite\{\}} Befehl eingefügt \cite{Tietze2010}.

%              [links]{rechts}
\begin{addmargin}[2em]{0pt}
    Dieser Text ist eingerückt. Hier könnte super ein sehr langes wörtliches Zitat stehen oder sonst was, das Einrückung bedarf.
\end{addmargin}

\section{Ein Abschnitt!}
Taucht auch im Inhaltsverzeichnis auf!
\subsection{Ein untergeordneter Abschnitt}
Taucht auch im Inhaltsverzeichnis auf!
\subsubsection{Noch eine Stufe tiefer!}
Wird nicht mehr im Inhaltsverzeichnis angezeigt.
Doch es gibt auch
\paragraph{Ein Paragraf} sogar eine
\subparagraph{Unter-Paragraf} ist möglich!


\chapter{Bilder einfügen}

Zur Bildpositionierung schaut bitte hier rein: \url{https://www.overleaf.com/learn/latex/Positioning_images_and_tables} besser könnte ich es auch nicht erklären.

Bilder sind sehr einfach einzufügen.
Man kann direkt auf \cref{img:KatzeAusHogwarts1} verweisen, und muss nichts weiter machen!
Wohl angemerkt, zeigt \cref{img:KatzeAusHogwarts1} eine Katze mit einem Zauberhut, nach allgemeinen Wissen, \textit{muss} dies von Hogwarts kommen.

Es lässt sich auf \cref{img:KatzeAusHogwarts1,img:KatzeAusHogwarts2} verweisen

\begin{figure}[H]
    \includegraphics[width=0.75\textwidth]{hogwarts_cat.jpg}
    \centering
    \vspace{-5pt}
    \caption{Katze, frisch aus Hogwarts!}
    \label{img:KatzeAusHogwarts1}
\end{figure}

Man kann die Katze auch ganz lange ziehen:

\begin{figure}[H]
    \includegraphics[height=\textheight,width=\textwidth,keepaspectratio=false]{hogwarts_cat.jpg}
    \centering
    \vspace{-5pt}
    \caption{Lange Katze, frisch aus Hogwarts!}
    \label{img:KatzeAusHogwarts2}
\end{figure}

Wer lieber gleich vier Katzen haben möchte, kann diese Toll darstellen:

\begin{figure}[H]
    \begin{minipage}{0.5\textwidth}
        \centering
        \includegraphics[width=0.9\textwidth]{hogwarts_cat.jpg}
        \caption*{Katze 1, frisch aus Hogwarts!}
        \vspace{2ex}
    \end{minipage}
    \begin{minipage}{0.5\textwidth}
        \centering
        \includegraphics[width=0.9\textwidth]{hogwarts_cat.jpg}
        \caption*{Katze 2, frisch aus Hogwarts!}
        \vspace{2ex}
    \end{minipage}
    \begin{minipage}{0.5\textwidth}
        \centering
        \includegraphics[width=0.9\textwidth]{hogwarts_cat.jpg}
        \caption*{Katze 3, frisch aus Hogwarts!}
        \vspace{2ex}
    \end{minipage}
    \begin{minipage}{0.5\textwidth}
        \centering
        \includegraphics[width=0.9\textwidth]{hogwarts_cat.jpg}
        \caption*{Katze 4, frisch aus Hogwarts!}
        \vspace{2ex}
    \end{minipage}
\end{figure}

\chapter{Tabellen einfügen}
\section{Schöne Tabellen mit booktabs}
\section{Normale Tabellen}

\chapter{Aufzählungen}

\chapter{Formeln Einfügen}
\begin{align}
    a & = b + c     \\
      & = d + e + f \\
      & = g + h
\end{align}


\chapter{Tips und Tricks}
\anfz{Strg + Click} im PDF Viewer (in VSCode-Tab oder Browser), um zur \LaTeX-Codestelle zu springen. \\
Andersherum geht das auch, indem die stelle mit dem Cursor markiert wird und dann \anfz{Strg + Alt + J} gedrückt wird. \\
Für den Fall, dass nicht an die richtige Stelle gesprungen wird, nochmal compilen.

Per Rechtsklick in VSCode, kann über "Format Document" die gesamte Datei einheitlich eingerückt werden.