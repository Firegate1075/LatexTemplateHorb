\chapter{Kapitel im Inhaltsverzeichnis}
Dieses Kapitel ist im Inhaltsverzeichnis sichtbar.

\chapter*{Kapitel nicht im Inhaltsverzeichnis}
Dieses Kapitel ist nicht im Inhaltsverzeichnis sichtbar und wird nicht nummeriert.

\chapter{Text schreiben}
Text wird einfach geschrieben.
Er kann auch \textbf{fett} oder \textit{kursiv} oder \textit{\textbf{beides}} sein.

Sogar \textcolor{blue}{die Farbe} kann geändert werden!

Mit einer Leerzeilen wird ein Absatz erstellt.
Mit \anfz{\textbackslash\textbackslash} wird ein\\
Zeilenumbruch erstellt.

Ganz einfach per \anfz{\textbackslash{}anfz\{Text\}} wird der Text in Anführungszeichen geschrieben.

Quellen werden mit einem \anfz{\textbackslash{}cite\{\}} Befehel eingefügt \cite{Tietze2010}.

\section{Ein Abschnitt!}
Tauch auch im Inhaltsverzeichnis auf!
\subsection{Ein untergeordneter Abschnitt}
Tauch auch im Inhaltsverzeichnis auf!
\subsubsection{Noch eine Stufe tiefer!}
Wird nicht mehr im Inhaltsverzeichnis angezeigt.
Doch es gibt auch
\paragraph{Ein Paragraf} sogar einen
\subparagraph{Unter-Paragrafen}