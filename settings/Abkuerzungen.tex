% \acro{Abk}{Abkürzung}
% \acro{ADS}{Automation Device Specification}
% \acro{AL}{Autonomous Life}
% \acro{API}{Application Programming Interface}
% \acro{AT}{Application Technology}
% \acro{AWL}{Anweisungsliste}
% \acro{BU}{Bussiness Unit}
% \acroplural{BU}[BUs]{Bussiness Units}
% \acro{CMD}{Command}
% \acro{CS}{Conveyor Systems}
% \acro{CSV}{Comma Seperated Values}
% \acro{DB}{Datenbaustein}
% \acro{DBS}{Datenbankserver}
% \acro{DHBW}{Duale Hochschule Baden-Württemberg}
% \acroplural{DHBW}[DHBW]{Dualen Hochschule Baden-Württemberg} %Genitiv
% \acro{ETA}{Electronic Travel Aid}
% \acro{FAD}{Feinabdichtung}
% \acro{FUP}{Funktionsplan}
% \acro{GAD}{Grobabdichtung}
% \acro{GDI}{General Device Interface}
% \acro{GUI}{Graphical User Interface}
% \acro{HMI}{Human-Machine-Interface}
% \acro{HRK}{Hohlraumkonservierung}
% \acro{IEC}{International Electrotechnical Commission}
% \acro{IDB}{Instanzdatenbaustein}
% \acro{IDE}{Integrated Development Environment}
% \acroplural{IDE}[IDEs]{Integrated Development Environments}
% \acro{IR}{Infrarot}
% \acro{ISO}{International Organization for Standardization}
% \acro{KTL}{Kataphoretische Tauchlackierung}
% \acro{IQ}{In-/Output}
% \acro{JSON}{JavaScript Object Notation}
% \acro{KTL}{Kataphoretische Tauchlackierung}
% \acro{LASD}{Liquid-Applied-Sound-Deadener}
% \acro{LED}{Light-Emitting-Diode}
% \acroplural{LED}[LEDs]{Light-Emitting-Diodes}
% \acro{MMS}{Mensch-Maschine-Schnittstelle}
% \acro{NVO}{Naht von oben (Sealingmethode)}
% \acro{NVU}{Naht von unten (Sealingmethode)}
% \acro{OPC-UA}{Open Plattform Communication - Unified Architecture}
% \acro{PAP}{Programmablaufplan}
% \acroplural{PAP}[PAP]{Programmablaufplans} %Genitiv
% \acro{PDE}{Process-Database-Editor}
% \acroplural{PDE}[PDE]{Process-Database-Editors}
% \acro{PDF}{Portable Document Format}
% \acro{PLC}{Programmable-Logic-Controller}
% \acroplural{PLC}[PLCs]{Programmable-logic-controllers}
% \acro{PLD}{Position Locator Device}
% \acro{ProfiNET}{Siemens-Bussystem}
% \acro{PUR}{Polyurethane (Kunststoffart)}
% \acro{PVC}{Polyvinylchlorid (Kunststoffart)}
% \acro{RC}{Robotercontroller}
% \acro{RGB}{Red, Green, Blue}
% \acro{ROS}{Robot Operating System}
% \acro{QM}{Qualitätsmanagement}
% \acro{SC}{Software and Controls}
% \acro{SCADA}{Supervisory Control and Data Acquisition}
% \acro{SCL}{Structured Control Language}
% \acro{SDK}{Software Development Kit}
% \acro{SE}{ Societas Europaea ((frz.) Unternehmensform)}
% \acro{SLAM}{Simultaneous Localization and Mapping}
% \acro{SPS}{Speicherprogrammierbare Steuerung}
% \acroplural{SPS}[SPSen]{speicherprogrammierbare Steuerungen}
% \acro{SPQReL}{Socialis robot PopulusQue Romanus et Lindensis}
% \acro{ST}{Struct}
% \acro{SUT}{System Under Test}
% \acro{TIA}{Siemens Step7 Totally Integrated Automation (Programm zur SPS-Programmierung nach EN61131)}
% \acro{UDT}{User-Defined Data-Types}
% \acro{UML}{Unified Modeling Language}
% \acro{VB}{Visual Basic}
% \acro{VDI}{Verein deutscher Ingenieure}
% \acro{VIBN}{Virtuelle Inbetriebnahme}
% \acroplural{VIBN}[VIBN]{virtuellen Inbetriebnahme} %Genitiv
% \acro{Visu}{Visualisierung}
% \acro{VW}{Volkswagen}
% \acro{XML}{Extensible Markup Language}

% \acro{DC}{Direct Current}
% \acro{RF}{Radio Frequency}
% \acro{DUT}{Device Under Test, deutsch: zu testender Baustein}
% \acro{RFIM}{\ac{RF} Interface Module, deutsch: RF-Schnittstellenmodul}
% \acro{RFIB}{\ac{RF} Interface Board, deutsch: RF-Schnittstellenplatine}
% \acro{NTC}{Negative Temperature Coefficient}
% \acro{PTC}{Positive Temperature Coefficient}
% \acro{DAU}{Data Acquisition Unit}
% \acro{LAN}{Local Area Network}
% \acro{LXI}{\ac{LAN} Extension for Instrumentation}
% \acro{LDO}{Low Dropout Regulator}

% Das Acro-Package wird verwendet.
% Zusätzlich unterstüzt diese Vorlage das Einfügen von Text zwischen der Abkürzung und
% ihrer Einführung bei der ersten Verwendung.
% Zum Beispiel:
% \ac{soc}[-Bausteine]
% Wird bei der ersten Verwendung zu:
% SOC-Bausteine (SOC: System-on-a-Chip)
% Bei weiterer Verwendung:
% SOC-Bausteine


\DeclareAcronym{pdf}{
    short = PDF,
    long = Portable Document Format
}

\DeclareAcronym{csv}{
    short = CSV,
    long = Comma-separated values
}

\DeclareAcronym{dc}{
    short = DC,
    long = Gleichstrom,
    foreign = direct current,
    foreign-babel = english,
    foreign-locale = english
}
\DeclareAcronym{rf}{
    short = RF,
    long = hochfrequenz,
    foreign = radio frequency,
    foreign-babel = english,
    foreign-locale = english
}
\DeclareAcronym{dut}{
    short = DUT,
    long = Prüfling,
    long-plural = e,
    foreign = Device Under Test,
    foreign-plural-form = Devices Under Test,
    foreign-babel = english,
    foreign-locale = english
}
\DeclareAcronym{ntc}{
    short = NTC,
    long = Heißleiter,
    long-plural-form = Heißleiter,
    foreign = Negative Temperature Coefficient Resistor,
    foreign-babel = english,
    foreign-locale = english
}
\DeclareAcronym{ptc}{
    short = PTC,
    long = Kaltleiter,
    long-plural-form = Kaltleiter,
    foreign = Positive Temperature Coefficient Resistor,
    foreign-babel = english,
    foreign-locale = english
}
\DeclareAcronym{lan}{
    short = LAN,
    long = Local Area Network,
    %foreign = Negative Temperature Coefficient,
    %foreign-babel = english,
    %foreign-locale = english
}
\DeclareAcronym{lxi}{
    short = LXI,
    long = Local-Area-Network-Erweiterung für Laborinstrumente,
    foreign = LAN Extension for Instrumentation,
    foreign-babel = english,
    foreign-locale = english
}
\DeclareAcronym{ldo}{
    short = LDO,
    long = Low-Drop Längsregler,
    long-plural =,
    foreign = Low Drop-Out Regulator,
    foreign-babel = english,
    foreign-locale = english
}

\DeclareAcronym{fpga}{
    short = FPGA,
    long = Field Programable Gate Array
}

\DeclareAcronym{fet}{
    short = FET,
    long = Feldeffekt-Transistor
}

\DeclareAcronym{gan}{
    short = GaN,
    long = Gallium-Nitrid
}

\DeclareAcronym{smd}{
    short = SMD,
    %long = surface-mounted device
    long = Oberflächenmontage,
    foreign = surface-mounted device,
    foreign-locale= english,
    foreign-babel = english
}

\DeclareAcronym{cul}{
    short = CuL,
    long = Kupferlack
}

\DeclareAcronym{soc}{
    short = SOC,
    long = System-on-a-Chip
}